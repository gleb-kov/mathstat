\documentclass{article}
\usepackage[T1]{fontenc}
\usepackage{amssymb}
\usepackage{listings}
\usepackage[utf8x]{inputenc}
\usepackage{amsmath,amssymb}
\usepackage{pdfpages}
\usepackage[russian,english]{babel}
\lstset{
    language=Octave,
    frame=single,
    inputencoding=utf8x,
    extendedchars=\true,
    texcl=\false,
    breaklines=true,
    breakatwhitespace=true,
    commentstyle={}
}
\usepackage[a4paper,top=1cm,bottom=2cm,left=1.5cm,right=1cm,marginparwidth=1.75cm]{geometry}

\makeatletter
\def\@seccntformat#1{
  \expandafter\ifx\csname c@#1\endcsname\c@section\else
  \csname the#1\endcsname\quad
  \fi}
\makeatother

\begin{document}

\selectlanguage{russian}
\title{Лабораторная работа 5, ТВМС}
\author{
	Бочарников Андрей, M3238\\
	Ковешников Глеб, M3238\\
	Шишкин Алексей, M3238
}
\maketitle

\begin{quote}
\selectlanguage{russian}
\section{Формулировка}
	Для трёх распределений $X \thicksim N(a, \sigma^2)$, $X \thicksim N(!!, !!)$ и распределения Лапласа, сравнить следующие оценки параметра $a$:\\
	1. \\
	2. \\
	3. \\
	!! Сравнивать оценки нужно с точки зрения квадратичного риска. сравнить их выборочные среднеквадратичные отклонения. Cравнить с теоретическими среднеквадратичными отклонениями.
\section{Входные данные}
        \begin{itemize}
            \item Объем выборки: $n_1 = 100, n_2 = 10000$
            \item $\alpha = 1$
	    \item ?? Параметры нормального распределения:
	    \item ?? Параметры равномерного распределения:
	    \item ?? Параметры распределения Лапласа: 
	    \item Количество выборок: $m = 100$
        \end{itemize}
\section{Программа 1}	
        Нормальное распределение. \\
\subsection{Исходный код}
	\lstinputlisting{task1.m}
\subsection{Выходные данные}

\section{Программа 2}
	Равномерное распределение. \\
\subsection{Исходный код}
	\lstinputlisting{task2.m}
\subsection{Выходные данные}

\section{Программа 3}
        Распределение Лапласа. \\
\subsection{Исходный код}
        \lstinputlisting{task3.m}
\subsection{Выходные данные}
	TODO: вывод о том какая из оценок с точки зрения квадратичного риска является наилучшей.
\section{Вывод}
\end{quote}
\end{document}

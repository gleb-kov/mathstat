\documentclass{article}
\usepackage[T1]{fontenc}
\usepackage{kpfonts}
\usepackage{amssymb}
\usepackage{listings}
\usepackage[utf8x]{inputenc}
\usepackage{amsmath,amssymb}
\usepackage{pdfpages}
\usepackage[russian,english]{babel}
\lstset{
    language=Octave,
    frame=single,
    inputencoding=utf8x,
    extendedchars=\true,
    texcl=\false,
    breaklines=true,
    breakatwhitespace=true,
    commentstyle={}
}
\usepackage[a4paper,top=1cm,bottom=2cm,left=1.5cm,right=1cm,marginparwidth=1.75cm]{geometry}

\makeatletter
\def\@seccntformat#1{
  \expandafter\ifx\csname c@#1\endcsname\c@section\else
  \csname the#1\endcsname\quad
  \fi}
\makeatother

\begin{document}

\selectlanguage{russian}
\title{ТВМС, Экзамен, Билет 1 (номер в ИСУ: 10)}
\author{
	Ковешников Глеб, M3238\\
	kovg16@gmail.com
}
\maketitle

\begin{quote}
\selectlanguage{russian}
\section{Вопрос 1}
\large
Вопрос: Функции потерь и функции риска, состоятельность оценки характеристики,
  достаточное условие для состоятельности оценки. Вид квадратичного риска в случае
  одномерной характеристики.\\

Оценка $\hat{g}(\theta)$ -- статистика вида: $\hat{g} \colon \mathcal{X} \to g(\Theta)$.

Функция потерь $l(\hat{g}, g(\theta))$ -- неотрицательная функция,
характеризующая близость оценки к реальному значению.

Принято считать функцию потерь неотрицательной монотонной функцией: \\
$l(\hat{g}, g(\theta)) = \omega({\lVert}\hat{g}, g(\theta){\rVert})$, где $\omega(0) = 0$.

Риск -- функция:
$R(\hat{g}, \theta) \overset{def}{\underset{}{=}} E_{\theta}(l(\hat{g}, g(\theta)))$

При асимптотическом подходе оценка называется состоятельной, если:\\
$\hat{g}_n \xrightarrow[n \to +\infty]{P_{n, \theta}} g(\theta)$

Квадратичная функция потерь:
$l_2(\hat{g}, g(\theta)) = {\lVert}{\hat{g}, g(\theta)}{\rVert}^2$

Квадратичный риск:
$R_2(\hat{g}, \theta) = E_\theta({\lVert}{\hat{g}, g(\theta)}{\rVert}^2)$

Определим функцию потерь индикатором отклонений:\\
$l^\delta(\hat{g}, g(\theta)) = \omega^\delta({\lVert}{\hat{g}, g(\theta)}{\rVert})$
\begin{equation*}
  \omega(t) = 1_\delta(t) = \begin{cases}
    0,~ t < \delta  \\
    1,~ t \geqslant \delta
  \end{cases}
\end{equation*}

Соответствующий риск будет вероятностью отклонения:\\
$
R^\delta(\hat{g}, \theta) = E_{\theta}(l(\hat{g}, g(\theta))) =
0 \cdot P_\theta({\lVert}{\hat{g}, g(\theta){\rVert}} < \delta) +
1 \cdot P_\theta({\lVert}{\hat{g}, g(\theta){\rVert}} \geqslant \delta) =
P_\theta({\lVert}{\hat{g}, g(\theta)}{\rVert} \geqslant \delta)
$\\

Теорема (достаточное условие для состоятельности оценки):\\
В случае одномерной оценки $R_2(\hat{g}_n, \theta) 
\xrightarrow[n \to +\infty]{} 0 \Longrightarrow$ оценка состоятельна.\\
Доказательство:\\
$
\forall \delta > 0~ R^\delta(\hat{g}_n, \theta)
= P({\lVert}{\hat{g}_n - g(\theta)}{\rVert} \geqslant \delta) =
P({\lVert}{\hat{g}_n - g(\theta)}^2{\rVert} \geqslant \delta^2)  \\
\leqslant \frac{E_\theta({\lVert}{\hat{g}_n - g(\theta)}^2{\rVert})}{\delta^2} =
\frac{R_2(\hat{g}_n, \theta)}{\delta^2} \xrightarrow[n \to +\infty]{} 0
$
\\ Ч.Т.Д. \\

Смещение оценки -- величина: $b(\hat{g}, \theta) = g(\theta) - E_\theta(\hat{g})$

Оценка несмещенная, если смещение равно $0$.\\

Теорема (вид квадратичного риска):\\
$R_2(\hat{g}, \theta) = D_\theta(\hat{g}) + b^2(\hat{g}, \theta)$ \\
Доказательство: \\
$
R_2(\hat{g}, \theta) = E_\theta({\lVert}{\hat{g} - g(\theta)}^2{\rVert}) \\
= E_\theta(\hat{g} - E_\theta(\hat{g}) - (g(\theta) - E_\theta(\hat{g})))^2 \\
= E_\theta(\hat{g} - E_\theta(\hat{g}))^2 + (g(\theta) -
E_\theta(\hat{g}))^2 - 2(g(\theta) - E_\theta(\hat{g}))(E_\theta{\hat{g}} - E_\theta{\hat{g}}) \\
= D_\theta(\hat{g}) + b^2(\hat{g}, \theta)
$ 
\\ Ч.Т.Д.\\

Для несмещенных оценок квадратичный риск равен дисперсии.
\section{Вопрос 2}
Вопрос: Гистограмма как оценка плотности распределения. Статистические свойства гистограммы. \\

Зададим плотность распределения в точке $t$ как
$f(t) \approx \frac{P(\Delta)}{|\Delta|}$, где
$\Delta = [t_1, t_2) \ni t$;
$t_1, t_2$ -- соседние точки разбиения отрезка $[a, b]$;
$|\Delta| = t_2 - t_1$.

Интервалы группировки -- разбиение
$\{\,\Delta_0, \Delta_{\pm 1}, \Delta_{\pm 2}, \ldots\,\}$ отрезка
$[a, b]$ на дизъюнктные интервалы фиксированной длины $> 0$.

Гистограмма -- функция $f_n(t)$, принимающая постоянные значения на заданных интервалах группировки:
$t \in \Delta_m \Longrightarrow f_{n}(t) = f_{n, m} = \frac{k(\Delta_m)}{nh}$,\\
где $k(\Delta)$ -- количество элементов выборки, лежащих в отрезке $\Delta$.

Гистограмма является кусочно-постоянной функцией.\\

Теорема. Гистограмма является плотностью распределения.\\
Доказательство:\\
1. $f_n(t) \geqslant 0$

2. $\int\limits_{\mathbb{R}}{f_n(t) dt}$
$= \sum_{m}{\int\limits_{\Delta_m}{f_{n}(t) dt}} =
        \sum_{m}{h f_{n, m}} = n^{-1} \sum_m{k(\Delta_m)} = 1$.
\\ Ч.Т.Д. \\

Теорема. Пусть задано абсолютно непрерывное распределение с плотностью $f(x)$,
отрезок $[a, b]$ и его разбиение с длинами интервалов $h_n$ такими,
чтобы выполнялись условия:\\
$h_n \xrightarrow[n \to +\infty]{} 0,~ nh_n \xrightarrow[n \to +\infty]{} +\infty$, \\
тогда соответствующая гистограмма является состоятельной оценкой плотности распределения.
\section{Задача}
Формулировка: найти ОМП параметра $u$ показательного распределения. \\
Решение:
\begin{equation*}
f(t, u) = 
 \begin{cases}
   1, $t < 0$ \\
   \frac{1}{u} \cdot \exp{\frac{-t}{u}}, $t >= 0$
 \end{cases}
\end{equation*}
$L(u, x_1, ..., x_n) = \prod_{i = 1}^n{f(x_i, u)} = \prod_{i = 1}^n{\frac{1}{u}\exp{\frac{-x_i}{u}}} =
  \frac{1}{u^n}\exp{\frac{-\sum_{i = 1}^{n}{x_i}}{u}}$

$l(u, x_1, ..., x_n) = \ln{\prod_{i = 1}^n{f(x_i, u)}} = -n\ln{(u)} - \frac{\sum_{i = 1}^{n}{x_i}}{u}$

$\frac{dl}{du} = \frac{-n}{u} + \frac{\sum_{i = 1}^{n}{x_i}}{u^2}$

$\frac{dl}{du} = 0 \Longrightarrow un = \sum_{i = 1}^{n}{x_i} \Longrightarrow u^{*}_n = \overline{X_n}$,
где $u^{*}_n$ -- стационарная точка логарифмической функции правдоподобия.

Проверим достаточное условие максимума (вторая производная $ < 0$): \\
$\frac{d^2l}{du^2}(\overline{X_n}) = \frac{n}{u^2} - \frac{2un}{u^3} = \frac{-n}{u^2} < 0$.

$\Longrightarrow u^{*}_n = \overline{X_n}$ -- точка максимума функции правдоподобия.\\
Ответ: ОМП параметра $u$ имеет вид $u^{*}_n = \overline{X_n}$.

\end{quote}
\end{document}
